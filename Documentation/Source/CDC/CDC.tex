
\section{But et motivation}
Ce projet s'inscrit dans le cadre du travail de Bachelor pour l'obtention du diplôme de Bachelor à l'\gls{EIA-FR} .

Le projet vise à mettre en place une solution d'intégration du système d'information de l'\gls{ESIB} permettant aux étudiants ainsi qu'aux professeurs de l'université de pouvoir accéder à l'information de manière personnalisée et à tout moment. 

Le potentiel des nouveaux appareils mobiles comme l'iPhone et l'iPad est énorme et il n'est jusqu'à présent pas exploité pour accéder au système d'information de l'\gls{ESIB} . A l'aide de ce projet, nous allons exploiter aux mieux les nouvelles technologies et permettre aux utilisateurs du campus de consulter simplement et rapidement des informations telles que le programme des cours, les menus des cafétérias, les notes des examens et le plan du campus.
\section{Objectifs}
\subsection{ Objectif Obligatoire}

\begin{itemize}
	\item \textbf{Priorité : 1}  Permettre aux étudiants et aux professeurs de s'identifier à l'aide de leurs Login et Password de l'université afin d'avoir un droit d'accès aux informations qui les concernent.
	\item \textbf{Priorité : 1} Afficher la carte du campus.
	\begin{enumerate}[a)]
		\item La position actuelle de l'utilisateur sera détectée à l'aide du \gls{GPS} de l'appareil et affichée sur la carte.
		\item L'utilisateur peut, à l'aide de la fonction `` chercher ``, trouver l'emplacement d'un cours, le bureau d'une personne ou le lieu d'un événement.
		\item Les informations de la carte sont enregistrées sur le serveur et peuvent être mises à jour à tout moment. Un système de cache évite de recharger la carte à chaque visite.
	\end{enumerate}
	\item \textbf{Priorité : 1} Permettre aux professeurs et aux étudiants d'afficher leurs horaires.
	\begin{enumerate}[a)]
		\item Quand on clique sur un cours, l'emplacement de ce dernier est affiché sur la carte.
		\item L'utilisateur peut sauvegarder son horaire sur l'appareil pour un accès offline.
	\end{enumerate}
	\item \textbf{Priorité : 2} Permettre aux étudiants de consulter le résultat des examens. \textbf{Cet objectif est conditionné par l'accord de l'administration et du services informatique de l'\gls{ESIB}.}
	\item \textbf{Priorité : 2} Permettre de consulter les nouvelles du campus.
		\begin{enumerate}[a)]
			\item Si une nouvelle est liée à un lieu, permettre de l'afficher facilement sur la carte.
		\end{enumerate}
	\item \textbf{Priorité : 2} Permettre l'accès à l'annuaire de l'université. 
		\begin{enumerate}[a)]
			\item Quand on clique sur un numéro de téléphone, l'appel est lancé.
			\item Quand on clique sur une adresse mail, la fenêtre d'envoi de mail de l'appareil s'ouvre.
		\end{enumerate}
\end{itemize}

\subsection{ Objectif Complémentaire}
Les fonctionnalités supplémentaires et facultatives sont décrites ici. Elles seront réalisées en cas d'avance sur le planning initial.
\begin{itemize}
	\item Afficher les menus des cafétérias.
	\item Afficher la carte du campus.
	\begin{enumerate}[a)]
		\item Permettre l'envoi d'un message : `` Où est-tu?  `` et le correspondant aura la possibilité de transmettre sa position actuelle qui sera affichée sur la carte du campus.
		\item Afficher l'emplacement d'une personne d'après son horaire.
	\end{enumerate}
	\item Permettre aux professeurs et aux étudiants d'afficher leurs horaires.
	\begin{enumerate}[a)]
		\item Permettre l'ajout de l'horaire dans le calendrier de l'appareil.
	\end{enumerate}
	\item Permettre de consulter les nouvelles du campus.
		\begin{enumerate}[a)]
			\item Permettre à l'administrateur d'ajouter les nouvelles depuis l'appareil 
		\end{enumerate}
	\item Permettre l'accès à l'annuaire de l'université 
		\begin{enumerate}[a)]
			\item Sauvegarder la personne trouvée en tant que nouveau contact sur l'appareil.
		\end{enumerate}
\end{itemize}
\section{Contraintes}
\begin{itemize}
	\item La solution doit être basée sur une approche intégrant l'utilisation de l'iPhone et l'iPad. L'application doit être compatible avec les deux appareils.
	\item La solution doit être paramétrable à l'aide de fichiers XML pour permettre l'utilisation sur un autre Campus.
	\item Le côté serveur de l'application et notamment les Web Services seront mis en place par le service informatique de l'\gls{ESIB}  
\end{itemize}


\section{Planning}
 \EPSFIGTEXTWIDTH{../comon/figures/timeLine.pdf}{Vue globale du planning du projet}{planGlob}

Le projet est divisé en trois grandes phases.
\begin{enumerate}
\item La première phase consiste à prendre en main le projet. Cette phase est aussi une phase d'apprentissage du développement \gls{Objective-C} qui est nécessaire pour la suite du projet. Les dépendances des éléments externes comme les Web Services seront aussi clarifiés dans cette phase.
\item La deuxième phases et une phase de réalisation. Cette phase est divisée en plusieurs itérations. Chaque itération est composée des étapes suivante: \textbf{Analyse, Spécification, Conception, Implémentation et Test}. Cette approche itérative a été choisie car elle diminue le risque d'échec du projet et permet une plus grande flexibilité. Le principe est de se concentrer à chaque itération sur une tâche précise. Le client, en l'occurrence l'\gls{ESIB}, validera le résultat après chaque itération, si celui-ci correspond bien à ses attentes.
\item La troisième et dernière phase consiste à préparé l'application pour qu'elle soit prête pour un déploiement sur l'App store et à finir la documentation.
\end{enumerate}
Une planification plus détaillée ainsi qu'une explication du cycle de développement sera effectuée dans le document \gls{SPMP}