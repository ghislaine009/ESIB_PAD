\section{Introduction}
	Ce chapitre explique une partie importante d'une application qui est celle de test. La section cas de test décrit les cas de test pour les tests de fonctionnalités. La partie protocole de test regroupe les protocoles de test qui ont été effectués après chaque release. 
	\subsection{Philosophie de test}
	\EPSFIGSCALE[1]{../comon/figures/testApercu.pdf}{Vue globale de l'architecture du système}{archGlob}
	Les tests se feront sur trois niveaux, le premier niveau est celui des tests de fonctionnalités qui sont des tests faits par l'humain selon un procédé décrit dans la section ~\ref{tc}.  Le deuxième niveau est celui fait dans le code et qui est reproductible automatiquement (Unit test).Le troisième niveau est celui des tests de fuite dans la mémoire(Memory Leaks), qui consiste à observer la mémoire lors de l'utilisation de l'application(cas d'utilisation chapitre~\ref{tc}) et de vérifier qu'aucune variable n'est stockée en mémoire indéfiniment. Il est important d'indiquer que les cas de tests sont imaginés en même temps que la spécification, ce qui nous permet d'avoir un point de référence concernant les objectifs à atteindre. Voir le \gls{SPMP} chapitre ''Processus technique'' pour plus d'information.\\
	 A la fin de chaque itération, un protocole de test est rédigé après avoir testé les nouvelles fonctionnalités et retesté les anciennes. Grâce à cette stratégie, on est sûr que les nouvelles fonctionnalités n'empêche pas le fonctionnement des anciennes et que le tout reste compatible.
\section{Organisation des tests }
	\subsection{Éléments à tester}
		\begin{itemize}
			\item Le bon fonctionnement des différents cas d'utilisations.
			\item Des tests unitaires pour la partie logique métier.
			\item Des analyses de fuites dans la mémoire (Leaks) doivent être faites, car pour être visible sur l'appstore, une application ne doit pas contenir de Leaks.
		\end{itemize}
	\subsection{Éléments à ne pas tester}
		\begin{itemize}
			\item La sécurité des web services.
			\item La cohérence des résultats retournés par les web services.
		\end{itemize}
	\subsection{Outils de test et environnement}
		Le framework OCUnit\footnote{\url{http://developer.apple.com/tools/unittest.html}} nous permet de faire écrire des tests unitaires. Son fonctionnement est assez proche de celui JUnit pour Java. Les fonctions tel que STAssertEqualObjects ou STAssertTrue permettent de faire échouer ou réussir un test.
		
		Pour le test des Memory Leaks l'outil Instruments\footnote{\url{http://developer.apple.com/library/mac/\#documentation/DeveloperTools/Conceptual/InstrumentsUserGuide}} qui est prévu à cet effet est utilisé. Depuis XCode, choisir 'Product \begin{math}
		\Rightarrow \end{math} Profile' pour exécuter à l'aide de Instruments. Une fois le code lancé, choisir Leaks comme template de base. Ensuite, utiliser l'application selon un cas de test. L'outil vous indiquera si des leaks existent et leurs emplacements.
		 
\section{Cas de test \label{tc}}
	\subsection{Navigation}
				 \begin{longtable}{m{4cm}|p{10cm}|}
				 \textbf{ ID} & 1 \\
				 \hline \textbf{Description} & Teste que l'on peut bien naviguer d'une vue à l'autre sans erreurs.\\
				 \hline \textbf{Déroulement} &
					 \begin{itemize}
						 \item Fermer complètement l'application si elle était ouverte.
						 \item Ouvrir l'application.
						 \item  Pour chaque vue, cliquer sur le logo pour ouvrir, voir si le résultat obtenu est cohérent, revenir au menu principal.
						 \item  Pour chaque vue, cliquer sur le logo pour ouvrir, modifier le contenu dans la vue, fermer l'application à l'aide du bouton menu de l'appareil, réouvrir l'application, vérifier que c'est toujours cette vue qui est visible et qu'aucune information n'a été perdue après la manœuvre. 
						 \item  Pour chaque vue, cliquer sur le logo pour ouvrir. faire 4 x une rotation de 90 degrés à l'appareil. Vérifier qu'après chaque rotation la vue est dans le bon sens et que les éléments sont affichés correctement. 
					 \end{itemize}
				 \\
			 \end{longtable} 
	\subsection{Paramétrage}
			 \begin{longtable}{m{4cm}|p{10cm}|}
			 \textbf{ ID} & 2 \\
			 \hline \textbf{Description} & Teste que l'on peut modifier les paramètres de l'application\\
			 \hline \textbf{Déroulement} &
				 \begin{itemize}
					 \item Fermer complètement l'application si elle était ouverte.
					 \item Ouvrir l'application.
					 \item  Ouvrir la fenêtre de paramètres.
					 \item  Pour chaque champ:
						 \begin{enumerate}
						 	\item Éditer la valeur.
						 	\item fermer la fenêtre de paramètres.
						 	\item Réouvrir la fenêtre de paramètres.
						 	\item Vérifier que les valeurs sont bien celles saisies auparavant.
						 \end{enumerate}
					 \item Modifier tout les paramètres.
					 \item Fermer complètement l'application(À l'aide du gestionnaire d'applications et non seulement à l'aide du bouton menu. )
					 \item Réouvrir l'application et être sûr que les modifications ont bien été enregistrées.
				 \end{itemize}
			 \\
		 \end{longtable} 
		\subsection{Carte}
					 \begin{longtable}{m{4cm}|p{10cm}|}
					 \textbf{ ID} & 3 \\
					 \hline \textbf{Description} & Test du bon fonctionnement de la carte.\\
					 \hline \textbf{Déroulement} &
						 \begin{itemize}
						  	\item Se rendre au campus de l'ESIB
						  	\item Se connecter à internet
							 \item Fermer complètement l'application si elle était ouverte.
							 \item Ouvrir l'application.
							 \item Ouvrir la fenêtre de la carte.
							 \item Presser le bouton Localisez-moi et vérifier que l'endroit retourné est au bon emplacement.
							 \item Vérifier que l'application affiche des indicateurs sur les principaux immeubles du campus.
							 \item Se déplacer dans  le campus est vérifier que l'indicateur de position actuelle suit le déplacement.
							\item Saisir le nom d'une personne dans la barre de recherche, vérifier qu'on obtient un indicateur concernant l'emplacement du bureau de cette personne en suivants les écrans.
							\item Saisir le nom d'une classe dans la barre de recherche et faire de même que l'étape précédente.
							\item Saisir le nom d'un bâtiment dans la barre de recherche et faire de même que l'étape précédente.
							\item Presser sur le bouton de Navigation par élément.
							\begin{itemize}
								\item Choisir: Bâtiments. 
								\item Choisir un bâtiment  spécifique et vérifier que son emplacement est affiché sur la carte. 
							\end{itemize}
							\item Se déconnecter d'Internet et recommencer les étapes précédentes. Les mêmes fonctionnalités doivent être visibles.
						 \end{itemize}
					 \\
				 \end{longtable} 
				 
		\subsection{News}
					 \begin{longtable}{m{4cm}|p{10cm}|}
					 \textbf{ ID} & 4 \\
					 \hline \textbf{Description} & Test du bon fonctionnement de l'affichage des news.\\
					 \hline \textbf{Déroulement} &
						 \begin{itemize}
						  	\item Installer à neuf l'application.
						  	\item Se connecter à internet.
							 \item Fermer complètement l'application si elle était ouverte.
							 \item Ouvrir l'application.
							 \item Ouvrir la fenêtre des news.
							 \item Comparer le résultat avec celui de la page internet:\url{http://www.usj.edu.lb/}
							 \item Visualiser le détail des news et vérifier que le détail correspond à la news.
							\item Se déconnecter d'Internet et recommencer les étapes précédentes. Les mêmes fonctionnalités doivent être visibles.
						 \end{itemize}
					 \\
				 \end{longtable} 

		\subsection{Directory}
					 \begin{longtable}{m{4cm}|p{10cm}|}
					 \textbf{ ID} & 5 \\
					 \hline \textbf{Description} & Test du bon fonctionnement de l'annuaire.\\
					 \hline \textbf{Déroulement} &
						 \begin{itemize}
						  	\item Installer à neuf l'application.
						  	\item Se connecter à internet.
							 \item Fermer complètement l'application si elle était ouverte.
							 \item Ouvrir l'application.
							 \item Ouvrir la fenêtre de l'annuaire.
							 \item Choisir au moins 3 filtres d'affichages différents et s'assurer que les résultats sont bien cohérents. 
							 \item Choisir une personne et envoyer un e-mail via le lien mail.
							 \item Sur l'iPhone, cliquer sur le numéro de téléphone d'une personne et vérifier que l'appel est lancé.
							\item Se déconnecter d'Internet et recommencer les étapes précédentes. Les mêmes fonctionnalités doivent être visibles.
						 \end{itemize}
					 \\
				 \end{longtable} 
				 
		\subsection{Calendrier}
					 \begin{longtable}{m{4cm}|p{10cm}|}
					 \textbf{ ID} & 6 \\
					 \hline \textbf{Description} & Test du bon fonctionnement de l'affichage du calendrier.\\
					 \hline \textbf{Déroulement} &
						 \begin{itemize}
						  	\item Installer à neuf l'application.
						  	\item Se connecter à internet.
							 \item Fermer complètement l'application si elle était ouverte.
							 \item Ouvrir l'application.
							 \item Paramétrer l'application pour obtenir les données depuis le web service local.
							 \item Ouvrir la fenêtre du calendrier.
							 \item Passez au jour suivant ou précédent en faisant un mouvement de glissement sur le calendrier.
							 \item En cliquant sur un cours, la carte s'affiche, avec l'emplacement de dernier. 
							\item Se déconnecter d'Internet et recommencer les étapes précédentes. Les mêmes fonctionnalités doivent être visibles.
						 \end{itemize}
					 \\
				 \end{longtable}

		\subsection{Résultats des examens}
					 \begin{longtable}{m{4cm}|p{10cm}|}
					 \textbf{ ID} & 7 \\
					 \hline \textbf{Description} & Test du bon fonctionnement de l'affichage des résultats des examens.\\
					 \hline \textbf{Déroulement} &
						 \begin{itemize}
						  	\item Installer à neuf l'application.
						  	\item Se connecter à internet.
							 \item Fermer complètement l'application si elle était ouverte.
							 \item Ouvrir l'application.
							 \item Paramétrer l'application pour obtenir les données depuis le web service local.
							 \item Ouvrir la fenêtre de résultats des examens.
							 \item Vérifier que les résultats sont affichés correctement et que les notes sont bien celles obtenues lors de l'examen.
							\item Se déconnecter d'Internet et recommencer les étapes précédentes. Les mêmes fonctionnalités doivent être visibles.
						 \end{itemize}
					 \\
				 \end{longtable} 
\section{Protocole de test}
	Ce qu'on entend par protocole de test, c'est l'exécution de l'application en suivant pas à pas les étapes décrites dans le cas de test. En parallèle à cette exécution, les test de Leaks dans la mémoire sont effectués.
	Après chaque release, tout le protocole de test est répété pour garantir que les nouvelles fonctionnalités n'empêche pas le bon fonctionnement des anciennes.
	L'exécution des testes unitaire fait aussi partie du protocole de test et leurs logs sont mentionnés ci dessous. 
		\subsection{Protocole de test 1}
		\textbf{Version testé:} 0.1 (\url{https://esibpad.googlecode.com/svn/tags/0.1}) \\
		\textbf{	Date du test :} 20/06/2011

		\subsubsection*{Cas de test : Navigation}
				 \begin{longtable}{m{4cm}|p{10cm}|}
				 \textbf{ ID} & 1 \\
				 \hline \textbf{Description} & Tester si l'on peut bien naviguer d'une vue à l'autre sans erreurs.\\
				 \hline \textbf{Commentaires} &Il n'existe pour le moment qu'une seule page. \\
				 \hline Objectif  atteint & {\color{green} Complètement 100\% \CheckedBox } \\
				\hline Visa & Elias Medawar \\	
				 \\
			 \end{longtable} 
 		\subsubsection*{Cas de test : Paramétrage}
		 \begin{longtable}{m{4cm}|p{10cm}|}
		 \textbf{ ID} & 2 \\
		 \hline \textbf{Description} & Tester si l'on peut modifier les paramètres de l'application\\
		 \hline \textbf{Commentaires} & 
		 	 	 \begin{enumerate}
				  		\item La fonction retenir n'est pas encore implémentée correctement, les valeurs sont de toutes façon enregistrées.
				  		\item La validité des champs n'est pas implémentée, les valeurs peuvent être incohérentes.
				  		\item Les valeurs des champs ''Retenir et carte'' ne sont enregistrées qu'en cas de modification d'un autre champ de type texte.
				  	\end{enumerate} \\
 				\hline Objectif atteint &  {\color{red}partiellement 75\% \XBox} \\
 				\hline Visa & Elias Medawar 	\\
		 \\
		 \end{longtable} 
		 \subsubsection*{Test unitaire}
		 \begin{lstlisting}[language=C,caption = Log des test unitaires]
Test Suite 'ESIB_PADTests' started at 2011-06-20 06:33:02 +0000
Test Case '-[ESIB_PADTests testSettings]' started.
 Testing the settings DAO
Test Case '-[ESIB_PADTests testSettings]' passed (0.003 seconds).
Test Suite 'ESIB_PADTests' finished at 2011-06-20 06:33:02 +0000.
Executed 1 test, with 0 failures (0 unexpected) in 0.003 (0.003) seconds
		 \end{lstlisting}
		Objectif atteint : {\color{green}Complètement 100 \% \CheckedBox}
		 \subsubsection*{Test de fuite dans la mémoire}
		 		 \EPSFIGTEXTWIDTH{../comon/figures/leeks_graph1.png}{Résultat de l'analyse des Leeks à l'aide d'Xcode}{leeks0.1}
		 Objectif atteint : {\color{green}Complètement 100 \% \CheckedBox}\\
		 On peut voir que le code ne contient aucune fuite de mémoire.
		 
		 
		\subsection{Protocole de test 2}
		 		\textbf{Version testé:} 0.2 (\url{https://github.com/eia-fr/ESIB_PAD/tree/0.2}) \\
		 		\textbf{	Date du test :} 05/07/2011
		 
		 		\subsubsection*{Cas de test : Navigation}
		 				 \begin{longtable}{m{4cm}|p{10cm}|}
		 				 \textbf{ ID} & 1 \\
		 				 \hline \textbf{Description} & Tester si l'on peut bien naviguer d'un vue à l'autre sans erreurs\\
		 				 \hline \textbf{Commentaires} &Lors du chargement des informations depuis internet, le logo loading n'est pas centré quand l'iphone est en paysage. \\
		 				 \hline Objectif  atteint & {\color{green} Complètement 100\% \CheckedBox } \\
		 				\hline Visa & Elias Medawar \\	
		 				 \\
		 			 \end{longtable} 
		  		\subsubsection*{Cas de test : Paramétrage}
		 		 \begin{longtable}{m{4cm}|p{10cm}|}
		 		 \textbf{ ID} & 2 \\
		 		 \hline \textbf{Description} & Tester si l'on peut modifier les paramètres de l'application\\
		 		 \hline \textbf{Commentaires} & 
		 		 	 	 \begin{enumerate}
		 				  		\item La fonction retenir n'est pas encore implémentée correctement, les valeurs sont de toutes façon enregistrées.
		 				  		\item La validité des champs est validée seulement au moment que les utiliser.
		 				  		\item La carte est toujours en mode satellite.
		 				  	\end{enumerate} \\
		  				\hline Objectif atteint &  {\color{red}partiellement 85\% \XBox} \\
		  				\hline Visa & Elias Medawar 	\\
		 		 \\
		 		  \end{longtable} 		 		 
		 		 \subsubsection*{Cas de test : Carte}
		 		 		 \begin{longtable}{m{4cm}|p{10cm}|}
		 		 		 \textbf{ ID} & 3 \\
		 		 		 \hline \textbf{Description} &  Test du bon fonctionnement de la carte.\\
		 		 		 \hline \textbf{Commentaires} &  
		 		 		 	 	 \begin{enumerate}
	 		 		 		 	 		\item La position de l'utilisateur est de toute façon affichée.
	 		 							\item Les coordonnées latitude et longitude sont inversées pour les batiments du campus CTS
	 		 							\item {\color{red}L'application ne fonctionne pas quand un campus n'as pas de bâtiment à afficher}.
	 		 		 		 	\end{enumerate} \\
	 		 		 		  				\hline Objectif atteint & {\color{orange} Partiellement 95\% \XBox } \\
	 		 		 		  				\hline Visa & Elias Medawar 	\\
		 		 		 \\
		 		 \end{longtable} 
		 		 \subsubsection*{Test unitaire}
		 		 \begin{lstlisting}[language=C,caption = Log des test unitaires]
Test Suite 'ESIB_PAD_SOURCESTests' started at 2011-07-05 07:37:38 +0000
Test Case '-[ESIB_PAD_SOURCESTests testCarte]' started.
 Testing the Campus DAO: You must uninstall the application before using this test
 Loading async the campus data from internet
 Waiting 30 sec for disabling the internet connection
 Getting campus data from cache?
 Comparing loacl and distant data
Test Case '-[ESIB_PAD_SOURCESTests testCarte]' passed (90.071 seconds).
Test Case '-[ESIB_PAD_SOURCESTests testSettings]' started.
 Testing the settings DAO
Test Case '-[ESIB_PAD_SOURCESTests testSettings]' passed (0.004 seconds).
Test Suite 'ESIB_PAD_SOURCESTests' finished at 2011-07-05 07:39:08 +0000.
Executed 2 tests, with 0 failures (0 unexpected) in 90.075 (90.077) seconds
		 		 \end{lstlisting}
		 		Objectif atteint : {\color{green}Complètement 100 \% \CheckedBox}\\
		 		\\
		 		Tester des appels de méthodes asynchrones n'est pas une tâche évidente. Notre classe de test tourne dans le thread A et la classe testée dans le thread B . Pour parvenir à faire cette opération on utilise les méthodes de synchronisation mises à disposition de l'IOS pour bloquer le thread A durant le temps du téléchargement des données depuis internet du thread B. Pour recevoir les données, notre classe de test est définie comme déléguée de la classe testée. Ansi quand le thread B  reçoit et a traité les données il notifie le thread A du résultat obtenu.
		 		 \subsubsection*{Test de fuite dans la mémoire}
		 		 	%\EPSFIGTEXTWIDTH{../comon/figures/leeks_graph1.png}{Résultat de l'analyse des Leeks à l'aide d'Xcode}{leeks0.1}
		 		 Objectif atteint : {\color{red}partiellement 50 \% \CheckedBox}\\
		 		 Un problème avec la classe NSPredicate crée des Leaks, selon la théorie le code n'en contient pas. Mais l'outil de mesure en détecte, des recherches plus approfondies pour trouver une solution seront réailsées.
		 
		\subsection{Protocole de test 3}
		 		\textbf{Version testé:} 0.3 (\url{https://github.com/eia-fr/ESIB_PAD/tree/0.3.1}) \\
		 		\textbf{	Date du test :} 14/07/2011
		 
		 		\subsubsection*{Cas de test : Navigation}
		 				 \begin{longtable}{m{4cm}|p{10cm}|}
		 				 \textbf{ ID} & 1 \\
		 				 \hline \textbf{Description} & Tester si l'on peut bien naviguer d'une vue à l'autre sans erreurs.\\
		 				 \hline \textbf{Commentaires} &Lors du chargement des informations depuis internet, le logo loading n'est pas centré quand l'iPhone		 est en paysage. \\
		 				 \hline Objectif  atteint & {\color{green} Complètement 100\% \CheckedBox } \\
		 				\hline Visa & Elias Medawar \\	
		 				 \\
		 			 \end{longtable} 
		  		\subsubsection*{Cas de test : Paramétrage}
		 		 \begin{longtable}{m{4cm}|p{10cm}|}
		 		 \textbf{ ID} & 2 \\
		 		 \hline \textbf{Description} & Tester si l'on peut modifier les paramètres de l'application.\\
		 		 \hline \textbf{Commentaires} & 
		 		 	 	 \begin{enumerate}
		 				  		\item La fonction retenir n'est pas encore implémentée correctement, les valeurs sont de toutes façon enregistrées.
		 				  	\end{enumerate} \\
		  				\hline Objectif atteint &  {\color{orange}partiellement 95\% \XBox} \\
		  				\hline Visa & Elias Medawar 	\\
		 		 \\
		 		  \end{longtable} 		 		 
		 		 \subsubsection*{Cas de test : Carte}
		 		 		 \begin{longtable}{m{4cm}|p{10cm}|}
		 		 		 \textbf{ ID} & 3 \\
		 		 		 \hline \textbf{Description} &  Test du bon fonctionnement de la carte.\\
		 		 		 \hline \textbf{Commentaires} &  
		 		 		 	 	 \begin{enumerate}
	 		 		 		 	 		\item La position de l'utilisateur est de toute façon affichée.
	 		 		 		 	\end{enumerate} \\
	 		 		 		  				\hline Objectif atteint & {\color{green} Complètement  100\% \CheckedBox } \\
	 		 		 		  				\hline Visa & Elias Medawar 	\\
		 		 		 \\
		 		 \end{longtable} 
		 		 \subsubsection*{Cas de test : News}
		 		 		 		 		 \begin{longtable}{m{4cm}|p{10cm}|}
		 		 		 		 		 \textbf{ ID} & 4 \\
		 		 		 		 		 \hline \textbf{Description} &  Test du bon fonctionnement de la carte.\\
		 		 		 		 		 \hline \textbf{Commentaires} &  
		 		 		 		 		 	 	 \begin{enumerate}
		 		 	 		 		 		 	 		\item Les news ne sont pas toujours les mêmes que celles sur le site, apparemment les webservices ne fournissent qu'une partie des news pour l'application.
		 		 	 		 					 	\end{enumerate} \\
		 		 	 		 		 		  				\hline Objectif atteint & {\color{green} Complètement 100\% \CheckedBox } \\
		 		 	 		 		 		  				\hline Visa & Elias Medawar 	\\
		 		 		 		 		 \\
		 		 		 		 \end{longtable} 
		 		 \subsubsection*{Test unitaire}
		 		 \begin{lstlisting}[language=C,caption = Log des test unitaires]
Test Suite 'ESIB_PAD_SOURCESTests' started at 2011-07-12 14:40:32 +0000
Test Case '-[ESIB_PAD_SOURCESTests testCarte]' started.
 Testing the Campus DAO: You must uninstall or reset cache of the application before testing
 Loading async the campus data from internet
 DATA recieved
Test Case '-[ESIB_PAD_SOURCESTests testCarte]' passed (60.564 seconds).
Test Case '-[ESIB_PAD_SOURCESTests testNews]' started.
 Testing the News DAO: You must uninstall or reset cache of the application before testing
 Loading async the news data from internet
 DATA recieved
Test Case '-[ESIB_PAD_SOURCESTests testNews]' passed (61.191 seconds).
Test Case '-[ESIB_PAD_SOURCESTests testSettings]' started.
Testing the settings DAO
Test Case '-[ESIB_PAD_SOURCESTests testSettings]' passed (0.073 seconds).
Test Suite 'ESIB_PAD_SOURCESTests' finished at 2011-07-12 14:42:34 +0000.
Executed 3 tests, with 0 failures (0 unexpected) in 121.828 (121.830) seconds
		 		 \end{lstlisting}
		 		Objectif atteint : {\color{green}Complètement 100 \% \CheckedBox}\\
		 		\\
		 		Le principe de test de classe asynchrone décrit plus haut est réutilisé pour le test: testNews.
		 		 \subsubsection*{Test de fuite dans la mémoire}
		 		 \EPSFIGTEXTWIDTH{../comon/figures/leeks_graph1.png}{Résultat de l'analyse des Leaks à l'aide d'Xcode}{leeks0.1}
		 		 Objectif atteint : {\color{green}complètement 100 \% \CheckedBox}\\
	
	
	
		\subsection{Protocole de test 4}
		 		\textbf{Version testé:} 0.5 (\url{https://github.com/eia-fr/ESIB_PAD/tree/05}) \\
		 		\textbf{	Date du test :} 05/08/2011
		 
		 		\subsubsection*{Cas de test : Navigation}
		 				 \begin{longtable}{m{4cm}|p{10cm}|}
		 				 \textbf{ ID} & 1 \\
		 				 \hline \textbf{Description} & Tester si l'on peut bien naviguer d'une vue à l'autre sans erreurs.\\
		 				 \hline \textbf{Commentaires} & Si on est entrain de loader des données depuis internet et que l'on choisit de passer à une autre fenêtre, l'application plante. \\
		 				 \hline Objectif  atteint & {\color{orange} Partiellement 95\% \CheckedBox } \\
		 				\hline Visa & Elias Medawar \\	
		 				 \\
		 			 \end{longtable} 
		  		\subsubsection*{Cas de test : Paramétrage}
		 		 \begin{longtable}{m{4cm}|p{10cm}|}
		 		 \textbf{ ID} & 2 \\
		 		 \hline \textbf{Description} & Tester si l'on peut modifier les paramètres de l'application\\
		  				\hline Objectif atteint &  {\color{green}Complètement 100\% \CheckedBox} \\
		  				\hline Visa & Elias Medawar 	\\
		 		 \\
		 		  \end{longtable} 		 		 
		 		 \subsubsection*{Cas de test : Carte}
		 		 		 \begin{longtable}{m{4cm}|p{10cm}|}
		 		 		 \textbf{ ID} & 3 \\
		 		 		 \hline \textbf{Description} &  Test du bon fonctionnement de la carte.\\
		 		 		 \hline \textbf{Commentaires} &  
		 		 		 	 	 \begin{enumerate}
	 		 		 		 	 		\item La position de l'utilisateur est de toute façon affichée.
	 		 		 		 	\end{enumerate} \\
	 		 		 	\hline Objectif atteint & {\color{green} Complètement  100\% \CheckedBox } \\
	 		 		 	\hline Visa & Elias Medawar 	\\
		 		 		 \\
		 		 \end{longtable} 
		 		 \subsubsection*{Cas de test : News}
		 		 		 		 		 \begin{longtable}{m{4cm}|p{10cm}|}
		 		 		 		 		 \textbf{ ID} & 4 \\
		 		 		 		 		 \hline \textbf{Description} &  Test du bon fonctionnement de la carte.\\
		 		 		 		 		 \hline \textbf{Commentaires} &  
		 		 		 		 		 	 	 \begin{enumerate}
		 		 	 		 		 		 	 		\item Les news ne sont pas toujours les mêmes que celles sur le site, apparemment les webservices ne fournissent qu'une partie des news pour l'application.
		 		 	 		 					 	\end{enumerate} \\
		 		 	 		 		 		 \hline Objectif atteint & {\color{green} Complètement 100\% \CheckedBox } \\
		 		 	 		 		 		\hline Visa & Elias Medawar 	\\
		 		 		 		 		\\
		 		 		 		 \end{longtable} 
		 		 		 		 
		 		 \subsubsection*{Cas de test : Directory}
		 		 		 		 		 \begin{longtable}{m{4cm}|p{10cm}|}
		 		 		 		 		 \textbf{ ID} & 5 \\
		 		 		 		 		 \hline \textbf{Description} &  Test du bon fonctionnement de l'annuaire.\\
		 		 		 		 		 \hline \textbf{Commentaires} &  
		 		 		 		 		 	 	 \begin{enumerate}
					 		 	 		 		 		 	 		\item Si on télécharge l'annuaire d'un Campus, et qu'ensuite on désire avoir l'annuaire d'une institution(qui est une sous-entité d'un campus) les informations de cette institution sont à nouveau téléchargées depuis internet. Ce qui signifie qu'on télécharge 2 fois les mêmes informations.
					 		 	 		 					 	\end{enumerate} \\
		 		 	 		 		 		 \hline Objectif atteint & {\color{orange} Partiellement 90\% \XBox } \\
					 		 	 		 		 		\hline Visa & Elias Medawar 	\\
					 		 		 		 		\\
		 		 		 		 \end{longtable}

		 		 \subsubsection*{Cas de test : Calendrier}
		 		 		 		 		 \begin{longtable}{m{4cm}|p{10cm}|}
		 		 		 		 		 \textbf{ ID} & 6 \\
		 		 		 		 		 \hline \textbf{Description} &  Test du bon fonctionnement de la carte.\\
		 		 		 		 		 \hline \textbf{Commentaires} &  
		 		 		 		 		 	 	 \begin{enumerate}
					 		 	 		 		 		 	 		\item aucun bug détecté.
					 		 	 		 					 	\end{enumerate} \\
		 		 	 		 		 		 \hline Objectif atteint & {\color{green} Complètement 100\% \CheckedBox } \\
					 		 	 		 		 		\hline Visa & Elias Medawar 	\\
					 		 		 		 		\\
		 		 		 		 \end{longtable} 


		 		 \subsubsection*{Test unitaire}
		 		 \begin{lstlisting}[language=C,caption = Log des test unitaires]
Test Suite 'ESIB_PAD_SOURCESTests' started at 2011-08-07 12:48:23 +0000
Test Case '-[ESIB_PAD_SOURCESTests testCarte]' started.
  Testing the Campus DAO: You must uninstall or reset cache of the application before testing
  Loading async the campus data from internet
  Recievied async the Campus DATA
Test Case '-[ESIB_PAD_SOURCESTests testCarte]' passed (2.025 seconds).
Test Case '-[ESIB_PAD_SOURCESTests testDirectory]' started.
  Testing the Person DAO for the directory fonctionality: You must uninstall or reset cache of the application before testing
  Loading async the directory of the cmapus CST from internet
  List of person recieved
Test Case '-[ESIB_PAD_SOURCESTests testDirectory]' passed (17.585 seconds).
Test Case '-[ESIB_PAD_SOURCESTests testNews]' started.
  Testing the News DAO: You must uninstall or reset cache of the application before testing
  Loading async the news data from internet
  Recievied async the news DATA
Test Case '-[ESIB_PAD_SOURCESTests testNews]' passed (4.727 seconds).
Test Case '-[ESIB_PAD_SOURCESTests testPlaning]' started.
  Testing the Horraire DAO for the planning  fonctionality: You must uninstall or reset cache of the application before testing
  Loading async the plannong from internet
 Recievied async the planning from internet
Test Case '-[ESIB_PAD_SOURCESTests testPlaning]' passed (18.140 seconds).
Test Case '-[ESIB_PAD_SOURCESTests testSettings]' started.
 Testing the settings DAO
Test Case '-[ESIB_PAD_SOURCESTests testSettings]' passed (0.005 seconds).
Test Suite 'ESIB_PAD_SOURCESTests' finished at 2011-08-07 12:49:25 +0000.
Executed 5 tests, with 0 failures (0 unexpected) in 55.508 (55.538) seconds
		 		 \end{lstlisting}
		 		Objectif atteint : {\color{green}Complètement 100 \% \CheckedBox}\\
		 		\\
		 		Le principe de test de classe asynchrone décrit plus haut est réutilisé pour les tests: testPlaning et testDirectory.
		 		 \subsubsection*{Test de fuite dans la mémoire}
		 		 \EPSFIGTEXTWIDTH{../comon/figures/leeks_graph1.png}{Résultat de l'analyse des Leaks à l'aide d'Xcode}{leeks0.1}
		 		 Objectif atteint : {\color{green}complètement 100 \% \CheckedBox}\\


		\subsection{Protocole de test 5}
		 		\textbf{Version testé:} 0.6 (\url{https://github.com/eia-fr/ESIB_PAD/tree/06}) \\
		 		\textbf{	Date du test :} 15/08/2011
		 
		 		\subsubsection*{Cas de test : Navigation}
		 				 \begin{longtable}{m{4cm}|p{10cm}|}
		 				 \textbf{ ID} & 1 \\
		 				 \hline \textbf{Description} & Tester si l'on peut bien naviguer d'une vue à l'autre sans erreurs.\\
		 				 \hline \textbf{Commentaires} & Dans la vue du calendrier,quand on ouvre la carte et après on fait une rotation de l'appareil, au retour au menu principal,  l'application plante. \\
		 				 \hline Objectif  atteint & {\color{orange} Partiellement 95\% \CheckedBox } \\
		 				\hline Visa & Elias Medawar \\	
		 				 \\
		 			 \end{longtable} 
		  		\subsubsection*{Cas de test : Paramétrage}
		 		 \begin{longtable}{m{4cm}|p{10cm}|}
		 		 \textbf{ ID} & 2 \\
		 		 \hline \textbf{Description} & Tester si l'on peut modifier les paramètres de l'application.\\
		  				\hline Objectif atteint &  {\color{green}Complètement 100\% \CheckedBox} \\
		  				\hline Visa & Elias Medawar 	\\
		 		 \\
		 		  \end{longtable} 		 		 
		 		 \subsubsection*{Cas de test : Carte}
		 		 		 \begin{longtable}{m{4cm}|p{10cm}|}
		 		 		 \textbf{ ID} & 3 \\
		 		 		 \hline \textbf{Description} &  Test du bon fonctionnement de la carte.\\
		 		 		 \hline \textbf{Commentaires} &  
		 		 		 	 	 \begin{enumerate}
	 		 		 		 	 		\item La position de l'utilisateur est de toute façon affichée.
	 		 		 		 	\end{enumerate} \\
	 		 		 	\hline Objectif atteint & {\color{green} Complètement  100\% \CheckedBox } \\
	 		 		 	\hline Visa & Elias Medawar 	\\
		 		 		 \\
		 		 \end{longtable} 
		 		 \subsubsection*{Cas de test : News}
		 		 		 		 		 \begin{longtable}{m{4cm}|p{10cm}|}
		 		 		 		 		 \textbf{ ID} & 4 \\
		 		 		 		 		 \hline \textbf{Description} &  Test du bon fonctionnement de la carte.\\
		 		 		 		 		 \hline \textbf{Commentaires} &  
		 		 		 		 		 	 	 \begin{enumerate}
		 		 	 		 		 		 	 		\item Les news ne sont pas toujours les mêmes que celles sur le site, apparemment les webservices ne fournissent qu'une partie des news pour l'application.
		 		 	 		 					 	\end{enumerate} \\
		 		 	 		 		 		 \hline Objectif atteint & {\color{green} Complètement 100\% \CheckedBox } \\
		 		 	 		 		 		\hline Visa & Elias Medawar 	\\
		 		 		 		 		\\
		 		 		 		 \end{longtable} 
		 		 		 		 
		 		 \subsubsection*{Cas de test : Directory}
		 		 		 		 		 \begin{longtable}{m{4cm}|p{10cm}|}
		 		 		 		 		 \textbf{ ID} & 5 \\
		 		 		 		 		 \hline \textbf{Description} &  Test du bon fonctionnement de l'annuaire.\\
		 		 		 		 		 \hline \textbf{Commentaires} &  
		 		 		 		 		 	 	 \begin{enumerate}
					 		 	 		 		 		 	 		\item Si on télécharge l'annuaire d'un Campus, et qu'ensuite on désire avoir l'annuaire d'une institution (qui est une sous-entité d'un campus) les informations de cette institution sont à nouveau téléchargées depuis internet. Ce qui signifie qu'on télécharge 2 fois les mêmes informations.
					 		 	 		 					 	\end{enumerate} \\
		 		 	 		 		 		 \hline Objectif atteint & {\color{orange} Partiellement 95\% \XBox } \\
					 		 	 		 		 		\hline Visa & Elias Medawar 	\\
					 		 		 		 		\\
		 		 		 		 \end{longtable}

		 		 \subsubsection*{Cas de test : Calendrier}
		 		 		 		 		 \begin{longtable}{m{4cm}|p{10cm}|}
		 		 		 		 		 \textbf{ ID} & 6 \\
		 		 		 		 		 \hline \textbf{Description} &  Test du bon fonctionnement du calendrier.\\
		 		 		 		 		 \hline \textbf{Commentaires} &  
		 		 		 		 		 	 	 \begin{enumerate}
					 		 	 		 		 		 	 		\item aucun bug détecté.
					 		 	 		 					 	\end{enumerate} \\
		 		 	 		 		 		 \hline Objectif atteint & {\color{green} Complètement 100\% \CheckedBox } \\
					 		 	 		 		 		\hline Visa & Elias Medawar 	\\
					 		 		 		 		\\
		 		 		 		 \end{longtable} 
		 		 \subsubsection*{Cas de test : Résultats des examens}
		 		 		 		 		 \begin{longtable}{m{4cm}|p{10cm}|}
		 		 		 		 		 \textbf{ ID} & 7 \\
		 		 		 		 		 \hline \textbf{Description} &  Test du bon fonctionnement de l'affichage des notes.\\
		 		 		 		 		 \hline \textbf{Commentaires} &  
		 		 		 		 		 	 	 \begin{enumerate}
		 		 					 		 	 		 		 		 	 		\item aucun bug détecté.
		 		 					 		 	 		 					 	\end{enumerate} \\
		 		 	 		 		 		 \hline Objectif atteint & {\color{green} Complètement 100\% \CheckedBox } \\
		 		 					 		 	 		 		 		\hline Visa & Elias Medawar 	\\
		 		 					 		 		 		 		\\
		 		 		 		 \end{longtable} 


		 		 \subsubsection*{Test unitaire}
		 		 \begin{lstlisting}[language=C,caption = Log des test unitaires]
Test Suite 'ESIB_PAD_SOURCESTests' started at 2011-08-07 12:48:23 +0000
Test Case '-[ESIB_PAD_SOURCESTests testCarte]' started.
  Testing the Campus DAO: You must uninstall or reset cache of the application before testing
  Loading async the campus data from internet
  Recievied async the Campus DATA
Test Case '-[ESIB_PAD_SOURCESTests testCarte]' passed (2.025 seconds).
Test Case '-[ESIB_PAD_SOURCESTests testDirectory]' started.
  Testing the Person DAO for the directory fonctionality: You must uninstall or reset cache of the application before testing
  Loading async the directory of the cmapus CST from internet
  List of person recieved
Test Case '-[ESIB_PAD_SOURCESTests testDirectory]' passed (17.585 seconds).
Test Case '-[ESIB_PAD_SOURCESTests testNews]' started.
  Testing the News DAO: You must uninstall or reset cache of the application before testing
  Loading async the news data from internet
  Recievied async the news DATA
Test Case '-[ESIB_PAD_SOURCESTests testNews]' passed (4.727 seconds).
Test Case '-[ESIB_PAD_SOURCESTests testPlaning]' started.
  Testing the Horraire DAO for the planning  fonctionality: You must uninstall or reset cache of the application before testing
  Loading async the plannong from internet
 Recievied async the planning from internet
Test Case '-[ESIB_PAD_SOURCESTests testPlaning]' passed (18.140 seconds).
Test Case '-[ESIB_PAD_SOURCESTests testSettings]' started.
 Testing the settings DAO
Test Case '-[ESIB_PAD_SOURCESTests testSettings]' passed (0.005 seconds).
Test Suite 'ESIB_PAD_SOURCESTests' finished at 2011-08-07 12:49:25 +0000.
Executed 5 tests, with 0 failures (0 unexpected) in 55.508 (55.538) seconds
		 		 \end{lstlisting}
		 		Objectif atteint : {\color{green}Complètement 100 \% \CheckedBox}\\
		 		\\

		 		 \subsubsection*{Test de fuite dans la mémoire}
		 		 \EPSFIGTEXTWIDTH{../comon/figures/leeks_graph1.png}{Résultat de l'analyse des Leaks à l'aide d'Xcode}{leeks0.1}
		 		 Objectif atteint : {\color{green}complètement 100 \% \CheckedBox}\\