
\newglossaryentry{ESIB}{
name={ESIB},
description={École Supérieure des Ingénieurs de Beyrouth- Faculté de l'USJ - Liban(\url{http://www.fi.usj.edu.lb/})}}
\newglossaryentry{USJ}{
name={USJ},
description={Université Saint-Joseph à Beyrouth. 5 campus dont l'FI,1873 enseignants,500 membres du personnel et
12000 étudiants(\url{http://www.usj.edu.lb/})}}

\newglossaryentry{EIA-FR}{
name={EIA-FR},
description={École d'ingénieurs et d'architectes de Fribourg- Suisse(\url{http://eia-fr.ch})}}
\newglossaryentry{GPS}{
name={GPS},
description={Le Global Positioning System (GPS) – que l'on peut traduire en français par « système de positionnement mondial » – est un système de géolocalisation fonctionnant au niveau mondial.\href{http://fr.wikipedia.org/wiki/Global\_Positioning\_System}{Plus de détail sur wikipedia} }}

\newglossaryentry{SPMP}{name={SPMP},
description={Software Project Management Plan est le doucment contenant toutes les informations concernant l'organisation d'un projet de développement de software selon la norme IEEE 1058 .\href{http://standards.ieee.org/findstds/standard/1058-1998.html}{Norme disponible à cette adresse}:\url{http://standards.ieee.org/findstds/standard/1058-1998.html}
 }
}

\newglossaryentry{Objective-C}
{name={Objective-C},
description={L'Objective-C est un langage de programmation orienté objet réflexif. C'est une extension du C ANSI, comme le C++, mais qui se distingue de ce dernier par sa distribution dynamique des messages, son typage faible ou fort, son typage dynamique et son chargement dynamique.Aujourd'hui, il est principalement utilisé pour le dévelopement d'application  Mac OS X et son dérivé iOS pou le développement iPhone,iPad,iPod.(Source wikipedia).  .\href{http://developer.apple.com/documentation/Cocoa/Conceptual/ObjectiveC/ObjC.pdf}{Référence Apple sur l'objective-c} :\url{http://developer.apple.com/documentation/Cocoa/Conceptual/ObjectiveC/ObjC.pdf}
}
}
\newglossaryentry{iOS}
{name={iOS},
description={iOS, anciennement iPhone OS, est le système d'exploitation mobile développé par Apple pour l'iPhone, l'iPod touch, et l'iPad..(Source wikipedia).\url{http://fr.wikipedia.org/wiki/IOS\_(Apple)}
}
}

\newglossaryentry{Skype}
{name={Skype},
description={Skype est un logiciel propriétaire qui permet aux utilisateurs de passer des appels téléphoniques via Internet. .   .\href{http://www.skype.com}{Site officiel} :\url{www.skype.com}
}
}

\newglossaryentry{SVN}
{name={SVN},
description={Subversion (en abrégé svn) est un système de gestion de versions, distribué sous licence Apache et BSD. \href{http://subversion.apache.org/}{Site officiel} :\url{http://subversion.apache.org/}
}
}

\newglossaryentry{Git}
{name={Git},
description={Git est un logiciel de gestion de versions décentralisée. C'est un logiciel libre créé par Linus Torvalds, le créateur du noyau Linux, et distribué sous la GNU GPL version 2.\url{ http://fr.wikipedia.org/wiki/Git}} 
}

\newglossaryentry{SRS}{name={SRS},
description={Software Requirements Specification(IEEE 830). Ce document contient la documentation concernant la spécification et l'analyse.
 }
}

\newglossaryentry{SDD}{name={SDD},
description={Software Design Description(IEEE 1016). Ce document contient la documentation concernant la conception  et l'implémentation
 }
}
\newglossaryentry{STD}{name={STD},
description={Software Test Documentation(IEEE 1016). Ce document contient la documentation concernant les tests effectué.
 }
}

\newglossaryentry{Web service}
{name={Web service},
description={Un service web est un programme informatique permettant la communication et l'échange de données entre applications et systèmes hétérogènes dans des environnements distribués. Il s'agit donc d'un ensemble de fonctionnalités exposées sur internet ou sur un intranet, par et pour des applications ou machines, sans intervention humaine, et de manière synchrone.(Source wikipedia)\url{http://fr.wikipedia.org/wiki/Service_Web}
}
}


\newglossaryentry{XCode}
{name={XCode},
description={XCode est un environnement de développement pour Mac OS X.\url{http://fr.wikipedia.org/wiki/Xcode}
}
}
\newglossaryentry{Core Data}
{name={Core Data},
description={Core Data is part of the Cocoa API in Mac OS X first introduced with Mac OS X 10.4 Tiger and for iOS with iPhone SDK 3.0.[2] It allows data organised by the relational entity-attribute model to be serialised into XML, binary, or SQLite stores. 
.(Source wikipedia)\url{http://en.wikipedia.org/wiki/Core\_Data}
}
}
